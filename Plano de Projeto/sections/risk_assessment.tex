\section{Avaliação de Riscos}

\subsection{Análise de Riscos}

O principal risco ao qual este projeto está sujeito é o atraso no seu desenvolvimento, fazendo com que não seja entregue nos prazos estabelecidos.

Riscos específicos identificados:

\begin{enumerate}
    \item \textbf{Risco Temporal:} Atrasos no cronograma devido à complexidade técnica
    \item \textbf{Risco de Recursos:} Indisponibilidade de membros da equipe
    \item \textbf{Risco Técnico:} Dificuldades na integração com sistemas externos
    \item \textbf{Risco de Escopo:} Mudanças de requisitos durante o desenvolvimento
    \item \textbf{Risco de Qualidade:} Falhas na validação e testes do sistema
    \item \textbf{Risco de Comunicação:} Falhas na coordenação entre equipes multidisciplinares
    \item \textbf{Risco Tecnológico:} Obsolescência ou incompatibilidade de ferramentas
\end{enumerate}

\subsection{Gerenciamento de Riscos}

Para mitigar os riscos mencionados, os gerentes devem fazer uma estimativa precisa de cronograma que englobe todas as atividades necessárias que farão o software ser desenvolvido como um produto completo e de alta qualidade.

\subsubsection{Estratégias de Mitigação}

\begin{itemize}
    \item \textbf{Controle Rigoroso do Cronograma:} Como todas as tarefas estão no caminho crítico, qualquer atraso impacta diretamente a entrega final
    \item \textbf{Acompanhamento de Marcos:} Monitoramento semanal do progresso de cada sprint
    \item \textbf{Recursos de Contingência:} Identificação de recursos alternativos para situações críticas
    \item \textbf{Comunicação Contínua:} Reuniões regulares para identificar problemas precocemente
    \item \textbf{Testes Incrementais:} Validação contínua para detectar problemas de qualidade
    \item \textbf{Documentação Abrangente:} Registro detalhado de decisões e mudanças
    \item \textbf{Treinamento da Equipe:} Capacitação em tecnologias e metodologias utilizadas
\end{itemize}

Para isso, este documento será usado para registrar métricas, estimar o tempo associado a cada atividade particular e organizar a interdependência entre atividades e a ordem em que elas serão realizadas pela equipe.

Dada essa informação, o tempo necessário para marcos particulares serem alcançados pode ser usado como sinal sobre se a equipe está mais lenta, mais rápida, ou na velocidade certa de desenvolvimento.

Esses sinais podem ser usados para refinar o conteúdo deste documento e instruir a equipe sobre como agir para entregar o produto final no prazo.

\subsubsection{Indicadores de Alerta}

\begin{itemize}
    \item Atraso superior a 2 dias em qualquer tarefa crítica
    \item Taxa de defeitos superior a 5\% nos testes de sprint
    \item Indisponibilidade de recursos por mais de 3 dias consecutivos
    \item Mudanças de escopo que impactem mais de 10\% do cronograma
    \item Problemas de comunicação entre equipes por mais de 24 horas
    \item Falhas recorrentes em builds ou deploys
\end{itemize}

\subsubsection{Plano de Contingência}

Em caso de materialização dos riscos principais:

\begin{enumerate}
    \item \textbf{Atraso no Cronograma:}
    \begin{itemize}
        \item Reavaliação de prioridades das funcionalidades
        \item Redistribuição de recursos entre equipes
        \item Extensão de jornada de trabalho (limitada)
        \item Simplificação de funcionalidades não críticas
    \end{itemize}
    
    \item \textbf{Problemas de Qualidade:}
    \begin{itemize}
        \item Intensificação dos testes pela equipe SQA
        \item Revisão de código por pares
        \item Implementação de ferramentas de análise estática
        \item Criação de ambiente de testes dedicado
    \end{itemize}
    
    \item \textbf{Indisponibilidade de Recursos:}
    \begin{itemize}
        \item Realocação temporária de membros entre equipes
        \item Contratação de consultor externo (se necessário)
        \item Redistribuição de tarefas entre membros disponíveis
        \item Priorização de atividades críticas
    \end{itemize}
\end{enumerate}

\subsubsection{Monitoramento e Controle}

\begin{itemize}
    \item \textbf{Reuniões Diárias:} Status de progresso e identificação rápida de problemas
    \item \textbf{Relatórios Semanais:} Análise de métricas e tendências
    \item \textbf{Revisões de Sprint:} Avaliação abrangente de qualidade e progresso
    \item \textbf{Escalação de Problemas:} Procedimentos claros para comunicação de riscos críticos
\end{itemize}
