\section{Resumo}

Este documento apresenta o plano completo para o desenvolvimento do Sistema de Gestão de Churrascos Entre Amigos, um software projetado para facilitar a organização, gestão financeira e coordenação de eventos sociais entre grupos de amigos.

O projeto será desenvolvido seguindo uma metodologia ágil com 4 sprints bem definidos, totalizando 86 dias úteis de desenvolvimento. O sistema contempla funcionalidades abrangentes desde o cadastro de usuários até a prestação de contas detalhada dos eventos.

\subsection{Objetivos Principais}
\begin{itemize}
    \item Automatizar o processo de organização de churrascos
    \item Facilitar o controle financeiro e divisão de despesas
    \item Implementar sistema de convites e confirmações automatizadas
    \item Proporcionar transparência na prestação de contas
    \item Ofertar cálculo automático de lista de compras
\end{itemize}

\subsection{Escopo do Projeto}
O sistema incluirá módulos para:
\begin{itemize}
    \item Gestão de usuários e autenticação
    \item Criação e gerenciamento de eventos
    \item Sistema de convites automatizado
    \item Controle de pagamentos e check-in
    \item Cálculo automático de lista de compras
    \item Sistema de prestação de contas
    \item Relatórios financeiros detalhados
\end{itemize}

\subsection{Cronograma Resumido}
\begin{itemize}
    \item \textbf{Sprint 0 (Preparação):} 27 dias - Documentação e planejamento
    \item \textbf{Sprint 1 (Desenvolvimento Inicial):} 26 dias - Cadastros e funcionalidades básicas
    \item \textbf{Sprint 2 (Desenvolvimento Avançado):} 23 dias - Sistema financeiro e cálculos
    \item \textbf{Sprint 3 (Finalização):} 10 dias - Testes finais e entrega
\end{itemize}

\textbf{Duração Total:} 86 dias úteis

\subsection{Equipe do Projeto}
O projeto será desenvolvido por uma equipe multidisciplinar composta por:
\begin{itemize}
    \item \textbf{Gerentes:} Julio, Ricardo, Igor
    \item \textbf{SQA (Garantia de Qualidade):} Luiz e Sara
    \item \textbf{A\&P (Análise e Projeto):} Augusto e Coleta
    \item \textbf{COD (Codificação):} Guilherme Carrara e Guilherme Digiorgi
\end{itemize}

\subsection{Principais Riscos Identificados}
\begin{itemize}
    \item Atrasos no cronograma devido à complexidade técnica
    \item Indisponibilidade de recursos humanos
    \item Mudanças de escopo durante o desenvolvimento
    \item Dificuldades de integração entre módulos
\end{itemize}

\subsection{Tecnologias e Metodologias}
O projeto utilizará uma abordagem ágil com foco em entregas incrementais, permitindo validação contínua dos requisitos e adaptação às necessidades do cliente. A arquitetura será modular para facilitar manutenção e escalabilidade futura.
