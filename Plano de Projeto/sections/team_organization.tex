\section{Organização da Equipe}

\subsection{Estrutura da Disciplina}

O projeto está inserido no contexto da disciplina de Engenharia de Software 2, onde cada dupla de alunos assume diferentes papéis em múltiplos projetos, proporcionando experiência completa no ciclo de desenvolvimento de software.

\subsection{Estrutura da Equipe para Este Projeto}

Para o \textbf{Projeto 3 - Sistema de Gestão de Churrascos Entre Amigos}, a estrutura da equipe segue o modelo de papéis rotacionados conforme especificado:

\begin{itemize}
    \item \textbf{Gerentes (G) - Julio, Ricardo, Igor:} 
    \begin{itemize}
        \item Planejamento estratégico e operacional
        \item Controle de cronograma e recursos
        \item Coordenação entre equipes das outras duplas
        \item Tomada de decisões de alto nível
        \item Comunicação com stakeholders e professor
        \item Criação e manutenção deste Plano de Projeto
        \item Monitoramento de riscos e qualidade
        \item Aprovação de entregáveis e marcos
    \end{itemize}
    
    \item \textbf{SQA - Garantia da Qualidade (Luiz e Sara):}
    \begin{itemize}
        \item Elaboração de planos de teste
        \item Execução de testes funcionais e não-funcionais
        \item Identificação e documentação de defeitos
        \item Revisão de artefatos de projeto
        \item Validação da qualidade final do produto
        \item Definição de critérios de aceitação
        \item Auditoria de processos de desenvolvimento
        \item Relatórios de qualidade e métricas
    \end{itemize}
    
    \item \textbf{A\&P - Analistas e Programadores (Augusto e Coleta):}
    \begin{itemize}
        \item Análise e especificação de requisitos
        \item Modelagem conceitual e de dados
        \item Design de arquitetura do sistema
        \item Criação de diagramas UML
        \item Validação de especificações
        \item Prototipação de interfaces
        \item Documentação técnica de análise
        \item Suporte ao design de banco de dados
    \end{itemize}
    
    \item \textbf{COD - Programadores (Guilherme Carrara e Guilherme Digiorgi):}
    \begin{itemize}
        \item Implementação das funcionalidades
        \item Codificação seguindo padrões estabelecidos
        \item Integração de módulos
        \item Correção de bugs e melhorias
        \item Documentação técnica do código
        \item Testes unitários básicos
        \item Configuração de ambiente de desenvolvimento
        \item Versionamento e controle de código
    \end{itemize}
\end{itemize}

\subsection{Matriz de Responsabilidades}

\begin{longtable}{|p{4cm}|c|c|c|c|}
\hline
\textbf{Atividade} & \textbf{Gerentes} & \textbf{SQA} & \textbf{A\&P} & \textbf{COD} \\
\hline
Planejamento & R & C & C & C \\
\hline
Análise de Requisitos & A & R & R & C \\
\hline
Design do Sistema & A & R & R & C \\
\hline
Codificação & A & C & C & R \\
\hline
Testes & A & R & C & C \\
\hline
Documentação & R & R & R & R \\
\hline
Integração & A & R & C & R \\
\hline
Entrega & R & A & C & C \\
\hline
\end{longtable}

\textbf{Legenda:} R = Responsável, A = Aprovador, C = Consultado

\subsection{Alocação por Sprint}

\subsubsection{Sprint 0 - Preparação}
\begin{itemize}
    \item \textbf{Gerentes (Julio, Ricardo e Igor):} Planejamento inicial e definição de diretrizes
    \item \textbf{A\&P (Augusto e Coleta):} Foco principal - documentação e modelagem
    \item \textbf{COD (Guilherme Carrara e Guilherme Digiorgi):} Preparação do ambiente de desenvolvimento
    \item \textbf{SQA (Luiz e Sara):} Planejamento de estratégias de teste e qualidade
\end{itemize}

\subsubsection{Sprint 1 - Desenvolvimento Inicial}
\begin{itemize}
    \item \textbf{Gerentes (Julio, Ricardo e Igor):} Acompanhamento e controle do cronograma
    \item \textbf{A\&P (Augusto e Coleta):} Finalização de diagramas de colaboração
    \item \textbf{COD (Guilherme Carrara e Guilherme Digiorgi):} Foco principal - codificação de cadastros
    \item \textbf{SQA (Luiz e Sara):} Testes e revisão do módulo de cadastro
\end{itemize}

\subsubsection{Sprint 2 - Desenvolvimento Avançado}
\begin{itemize}
    \item \textbf{Gerentes (Julio, Ricardo e Igor):} Monitoramento de riscos e ajustes no plano
    \item \textbf{A\&P (Augusto e Coleta):} Suporte à codificação complexa
    \item \textbf{COD (Guilherme Carrara e Guilherme Digiorgi):} Foco principal - sistema financeiro e cálculos
    \item \textbf{SQA (Luiz e Sara):} Testes extensivos do sistema financeiro
\end{itemize}

\subsubsection{Sprint 3 - Finalização}
\begin{itemize}
    \item \textbf{Gerentes (Julio, Ricardo e Igor):} Preparação da entrega e documentação final
    \item \textbf{A\&P (Augusto e Coleta):} Revisão final e documentação
    \item \textbf{COD (Guilherme Carrara e Guilherme Digiorgi):} Correção de bugs e refinamentos
    \item \textbf{SQA (Luiz e Sara):} Testes finais e validação completa do sistema
\end{itemize}

\subsection{Comunicação e Coordenação}

\subsubsection{Reuniões Regulares}
\begin{itemize}
    \item \textbf{Daily Standups:} Reuniões diárias de 15 minutos entre as duplas
    \item \textbf{Sprint Planning:} Planejamento detalhado no início de cada sprint
    \item \textbf{Sprint Review:} Avaliação dos resultados ao final de cada sprint
    \item \textbf{Retrospectivas:} Identificação de melhorias no processo
    \item \textbf{Reuniões Interdisciplinares:} Coordenação com outros projetos da disciplina
\end{itemize}

\subsubsection{Ferramentas de Gestão}
\begin{itemize}
    \item Sistema de controle de versão (Git)
    \item Ferramenta de gestão de projetos (ex: Jira, Trello)
    \item Plataforma de comunicação (ex: Slack, Teams, WhatsApp)
    \item Repositório de documentação centralizado
    \item Ambiente de desenvolvimento compartilhado
\end{itemize}

\subsection{Relatórios de Gestão}

\subsubsection{Estrutura de Reportes}

O projeto seguirá uma estrutura hierárquica de reportes para garantir comunicação eficaz e controle adequado:

\begin{itemize}
    \item \textbf{Relatórios Diários:} Status updates das tarefas em andamento
    \item \textbf{Relatórios Semanais:} Progresso detalhado por sprint e identificação de impedimentos
    \item \textbf{Relatórios de Marco:} Avaliação completa ao final de cada sprint
    \item \textbf{Relatórios de Exceção:} Comunicação imediata de problemas críticos
\end{itemize}

\subsubsection{Métricas de Acompanhamento}

\begin{enumerate}
    \item \textbf{Progresso de Cronograma:}
    \begin{itemize}
        \item Percentual de tarefas concluídas no prazo
        \item Variação entre estimado vs realizado
        \item Identificação de gargalos
    \end{itemize}
    
    \item \textbf{Qualidade:}
    \begin{itemize}
        \item Número de defeitos por módulo
        \item Taxa de retrabalho
        \item Cobertura de testes
    \end{itemize}
    
    \item \textbf{Recursos:}
    \begin{itemize}
        \item Utilização da equipe por perfil
        \item Disponibilidade vs demanda
        \item Identificação de sobrecarga
    \end{itemize}
\end{enumerate}

\subsection{Responsabilidades Específicas dos Gerentes}

Como líderes do Projeto 3, Julio, Ricardo e Igor têm as seguintes responsabilidades adicionais:

\begin{itemize}
    \item Coordenar atividades com as outras duplas envolvidas no projeto
    \item Reportar progresso para o professor da disciplina
    \item Gerenciar dependências entre projetos da disciplina
    \item Assegurar qualidade e cumprimento de prazos
    \item Facilitar comunicação entre diferentes perfis
    \item Tomar decisões estratégicas quando necessário
    \item Manter documentação atualizada e acessível
    \item Resolver conflitos e impedimentos
    \item Aprovar mudanças de escopo ou cronograma
\end{itemize}

\subsection{Critérios de Qualidade e Entrega}

\begin{itemize}
    \item \textbf{Código:} Seguir padrões de codificação definidos, comentários adequados, versionamento correto
    \item \textbf{Documentação:} Completa, atualizada e revisada por pelo menos dois membros da equipe
    \item \textbf{Testes:} Cobertura mínima de 80\%, testes automatizados onde possível
    \item \textbf{Integração:} Builds automatizados sem falhas, integração contínua funcional
\end{itemize}
